%%%%%%%%%%%%%%%%%%%%%%%%%%%%%%%%%%%%%%%
% Deedy - One Page Two Column Resume
% LaTeX Template
% Version 1.1 (30/4/2014)
%
% Original author:
% Debarghya Das (http://debarghyadas.com)
%
% Original repository:
% https://github.com/deedydas/Deedy-Resume
%
% IMPORTANT: THIS TEMPLATE NEEDS TO BE COMPILED WITH XeLaTeX
%
% This template uses several fonts not included with Windows/Linux by
% default. If you get compilation errors saying a font is missing, find the line
% on which the font is used and either change it to a font included with your
% operating system or comment the line out to use the default font.
% 
%%%%%%%%%%%%%%%%%%%%%%%%%%%%%%%%%%%%%%
% 
% TODO:
% 1. Integrate biber/bibtex for article citation under publications.
% 2. Figure out a smoother way for the document to flow onto the next page.
% 3. Add styling information for a "Projects/Hacks" section.
% 4. Add location/address information
% 5. Merge OpenFont and MacFonts as a single sty with options.
% 
%%%%%%%%%%%%%%%%%%%%%%%%%%%%%%%%%%%%%%
%
% CHANGELOG:
% v1.1:
% 1. Fixed several compilation bugs with \renewcommand
% 2. Got Open-source fonts (Windows/Linux support)
% 3. Added Last Updated
% 4. Move Title styling into .sty
% 5. Commented .sty file.
%
%%%%%%%%%%%%%%%%%%%%%%%%%%%%%%%%%%%%%%%
%
% Known Issues:
% 1. Overflows onto second page if any column's contents are more than the
% vertical limit
% 2. Hacky space on the first bullet point on the second column.
%
%%%%%%%%%%%%%%%%%%%%%%%%%%%%%%%%%%%%%%

\documentclass[]{deedy-resume-openfont}
\usepackage{fontawesome}
\usepackage{setspace}

\begin{document}

%%%%%%%%%%%%%%%%%%%%%%%%%%%%%%%%%%%%%%
%
%     TITLE NAME
%
%%%%%%%%%%%%%%%%%%%%%%%%%%%%%%%%%%%%%%


\namesection{Hamza}{Baig}{ \urlstyle{same}\url{www.hamzais.me} \\
 Electrical Engineering | University of Waterloo '22 | \href{mailto:mhabaig@uwaterloo.ca}{mhabaig@uwaterloo.ca}
}

%%%%%%%%%%%%%%%%%%%%%%%%%%%%%%%%%%%%%%
%
%     COLUMN ONE
%
%%%%%%%%%%%%%%%%%%%%%%%%%%%%%%%%%%%%%%

\begin{minipage}[t]{0.33\textwidth} 

%%%%%%%%%%%%%%%%%%%%%%%%%%%%%%%%%%%%%%
%     EDUCATION
%%%%%%%%%%%%%%%%%%%%%%%%%%%%%%%%%%%%%%

\section{Skills} 
\vspace{0.6em}
\descript{\begin{tightemize}{\item C++ \item C \item Python \item Assembly \item PowerShell \item Bash \item SQL \item Power Query \item LaTeX \item Markdown}}
\end{tightemize}
\sectionsep


%%%%%%%%%%%%%%%%%%%%%%%%%%%%%%%%%%%%%%
%     LINKS
%%%%%%%%%%%%%%%%%%%%%%%%%%%%%%%%%%%%%%

\section{Tools} 
\vspace{0.6em}
\descript{\begin{tightemize}
\item CLion \item Visual Studio  \item Linux \item Git \item Amazon Web Services \item Microsoft Azure \item Virtual Environments \item PowerBI
\end{tightemize}
}
\sectionsep

\section{Social}
{\faLinkedinSquare \, : \custombold{in/mhabaig/}} \\
\href{https://github.com/HBaig30}{\faGithubSquare \,: \custombold{HBaig30}} \\
\href{http://www.hamzais.me/}{\faGlobe\,:\,\custombold{www.hamzais.me}}

\sectionsep



%%%%%%%%%%%%%%%%%%%%%%%%%%%%%%%%%%%%%%
%
%     COLUMN TWO
%
%%%%%%%%%%%%%%%%%%%%%%%%%%%%%%%%%%%%%%

\end{minipage} 
\hfill
\begin{minipage}[t]{0.66\textwidth} 

%%%%%%%%%%%%%%%%%%%%%%%%%%%%%%%%%%%%%%
%     EXPERIENCE
%%%%%%%%%%%%%%%%%%%%%%%%%%%%%%%%%%%%%%

\section{Experience}
\vspace{0.5em}
\runsubsection{Virtual Research Student}
\\
{Communications Research Centre \textbullet{} Ottawa, ON \textbullet{} Sep 2018-Dec 2018}
\begin{tightemize}
\item Used AWS and Azure to develop and automate a program that determined idle virtual machines from active ones alongside giving users their real-time CPU usage 
\item Designed and implemented the program in Python and PowerShell, with data being stored in Azure’s virtual
database along with Amazon S3 buckets and curated using Microsoft’s Transact-SQL
\item Designed the front-end in Power BI which served the researchers through a web interface
\item Provided thorough documentation pertaining software design decisions and a troubleshooting guide using
Markdown to efficiently debug issues related to the process of data collection and filtering of virtual machines 

\end{tightemize}
\vspace{0.5em}
\runsubsection{Test Engineering Student}\\
{WindRiver Systems \textbullet{} Ottawa, ON \textbullet{} Jan 2018-Apr2018}
%\vspace{\topsep} % Hacky fix for awkward extra vertical space
\begin{tightemize}
\item Developed and implemented an auditing tool in Python which parsed through over 600 JSON files and
organized data into CSV format 
\item Resolved various bugs by debugging source code through disassembly files of various architectures such as
PowerPC and ARM 
\item Performed coverage-gap analysis on branch gaps existing in both source and disassembly files 

\end{tightemize}


%%%%%%%%%%%%%%%%%%%%%%%%%%%%%%%%%%%%%%
%     RESEARCH
%%%%%%%%%%%%%%%%%%%%%%%%%%%%%%%%%%%%%%

\section{Projects}
\vspace{0.5em}
\runsubsection{Robotic Arm}
\descript{Arduino,C}
\begin{tightemize}
\item Built and programmed a two-pronged robotic arm controlled by gyroscopic sensors with the use of an Arduino
and C to establish a platform for the interactions between the hardware and software \\
\item Developed a working knowledge of datasheets to make effective use of integrated circuits used in the project 
\end{tightemize}
\sectionsep

\runsubsection{www.HamzaIs.Me} {\href{https://github.com/HBaig30/hamzais.me}{{\faGithub}}
\descript{AWS,HTML,CSS,JavaScript}
\begin{tightemize}
\item Developed and hosted a personal website using Amazon Web Services  
\item Utilized both HTML/CSS along with JavaScript to provide users comfortable and user-friendly experience
\end{tightemize}
\sectionsep

\runsubsection{Magic Square}{\href{https://github.com/HBaig30/C-Projs_Coop_F18/blob/master/magicSquare.cpp}{{\faGithub}}
\descript{C++}
\begin{tightemize}
\item Utilized core concepts of number theory to develop a program in C++ to create any n-by-n grid where each
row, column and diagonal add up to the same number   \\
\end{tightemize}
\sectionsep

%%%%%%%%%%%%%%%%%%%%%%%%%%%%%%%%%%%%%%
%     SOCIETIES
%%%%%%%%%%%%%%%%%%%%%%%%%%%%%%%%%%%%%%
\section{Online Coursework}
\runsubsection{Introduction to Machine Learning}{\href{https://github.com/HBaig30/Markdown-Notes}{{\faGithub}} 
\begin{tightemize}
\item Learning the end-to-end process of investigating data through machine learning, using sklearn Python modules
\item Notes were written in Markdown, made available through GitHub
\end{tightemize}
\sectionsep

\end{minipage} 
\end{document}